%\documentclass{aportaciones}
%\usepackage[utopia,sfscaled]{mathdesign}
%\usepackage[scaled]{helvet}


%\aporta

\usepackage{setspace}
%\usepackage{helvet}
%\setstretch{0.98}
%\setstretch{.95}
\def\hb{\hfill\break}
\def\ctrcol#1{\centerline {\vbox {\halign {\hfil ##\hfil \cr #1\cr }}}}
\def\bibl#1{\hangindent=.6cm \noindent \hbox to .6cm{#1\hfill}} 
%-----------------------------------------------------------------------------

\begin{document}
\sffamily
\renewcommand{\familydefault}{\sfdefault}
\pagestyle{empty}
\thispagestyle{empty}
\parindent=0pt
\scriptsize

%$\phantom{.}$
%\vskip-36pt

\centerline{\large\bfseries PUBLICACIONES DEL}

\smallskip
\centerline{\large\bfseries INSTITUTO DE MATEMÁTICAS, UNAM}

\vskip5pt %10pt 
{\bfseries MONOGRAFÍAS DEL INSTITUTO DE MATEMÁTICAS}

\vskip3pt %5pt
\bibl{\hbox{\hskip3.5pt}1}{\itshape Localization in non commutative rings.} 
B.$\!$~J. Mueller (1975). %73~p. 

\bibl{\hbox{\hskip3.5pt}2}{\itshape Teoría de los números algebraicos.}
A. Díaz-Barriga, A.$\!$~I. Ramírez-Galarza,
F. Tomás (1975). %231~p.

\bibl{\hbox{\hskip3.5pt}3}{\itshape Integrales de medida positiva.} 
G. Zubieta (1976). %157~p.

\bibl{\hbox{\hskip3.5pt}4}{\itshape Rings with polynomial identities.} 
B.$\!$~J. Mueller (1977). %46~p.

\bibl{\hbox{\hskip3.5pt}5}{\itshape Grupos profinitos, grupos libres y productos libres.} 
L. Ribes (1977). %147~p.

\bibl{\hbox{\hskip3.5pt}6}{\itshape Introducción a la teoría de las clases
características en la geometría algebraica.} 
A. Holme (1978). %56~p.

\bibl{\hbox{\hskip3.5pt}7}{\itshape Embeddings, projective invariants and classifications.} 
A. Holme (1979). %122~p.

\bibl{\hbox{\hskip3.5pt}8}{\itshape The relative spectral sequence of Leray-Serre for
fibrations pairs.} 
C. Prieto (1979). %102~p.

\bibl{\hbox{\hskip3.5pt}9}{\itshape Caminatas aleatorias y movimiento browniano.} 
D.$\!$~B. Hernández (1981). %181~p.

\bibl{10}{\itshape On polarized varieties.} 
T. Matsusaka (1981). %38~p.

\bibl{11}{\itshape Métodos diagramáticos en teoría de
representaciones.} 
C. Cibils, F. Larrión, L. Salmerón (1982). %111~p.

\bibl{12}{\itshape Espacios simplécticos sobre $\beta$-anillos
y anillos de Hermite, trasvecciones simplécticas.}
E. Fernández Bermejo (1982). %68~p.

\bibl{13}{\itshape Abelian integrals.} 
G. Kempf (1983). %225~p.

\bibl{14}{\itshape On Selfinjective algebras of finite representation
type.} 
J. Waschb\"usch (1983). %58~p.

\bibl{15}{\itshape On ideal theory in Banach and topological algebras.} 
W. Zelazko (1984). %151~p.

\bibl{16}{\itshape Teoría general de procesos e integración
estocástica.} 
T. Bojdecki (1985). %105~p.

\bibl{17}{\itshape La figura espectral de operadores.} 
C. Hernández Garciadiego, E. de Oteyza (1986). %126~p.

\bibl{18}{\itshape Teoría de punto fijo.} 
A. Dold. Traducción: C. Prieto. 
Vol.~I, %(1986). 176~p. 
Vol.~II, %(1986). 215~p. 
Vol.~III (1986). %188~p.

\bibl{19}{\itshape Projective embeddings of algebraic varieties.}  
J. Roberts (1988). %84~p.

\bibl{20}{\itshape Teoría general de procesos e integración
estocástica.} 
2a. Edición.
T. Bojdecki (1989). %324~p.

\bibl{21}{\itshape Análisis funcional I.} 
C. Bosch Giral, E. Fernández Bermejo (1989). %95~p.

\bibl{22}{\itshape Introducción a la topología de las variedades de
dimensión infinita.} 
L. Montejano (1989). %95~p.

\bibl{23}{\itshape Introducción a la teoría de
representaciones de álgebras.} 
R. Martínez-Villa (1990). %97~p.

\vskip6pt %10pt 
{\bfseries Memorias del 50 aniversario del Instituto 
		de Matemáticas. 1942--1992.} %364~p.

\vskip6pt %10pt
{\bfseries Matemáticas en la UNAM. Memorias del 60
         Aniversario del Instituto de Matemáticas.} (2003). %127~p.

\vskip6pt %10pt 
{\itshape Topología algebraica. Un enfoque homotópico.}
M. Aguilar, S. Gitler, C. Prieto. Coedición de\\
\hbox{\hskip.6cm} Instituto de Matemáticas, UNAM -- McGraw-Hill
Interamericana Editores (1998). %493~p. 

%\vskip6pt
%{\itshape Sumario compendioso de las cuentas de plata y oro que en los reinos del Perú son necesarias a los\hfill\break 
%\hbox{\hskip.6cm} mercaderes y a todo género de tratantes. Con algunas reglas tocantes a la Artimética.}
%Hecho\hfill\break 
%\hbox{\hskip.6cm} por Juan Díez Freyle. 
%Edición facsimilar. Bibliotheca Mexicana Historiae Scientiarum. 
%Estudio\hfill\break 
%\hbox{\hskip.6cm} histórico de Marco Arturo Romero Corral. 
%Análisis matemático de César Guevara Bravo.\hfill\break 
%\hbox{\hskip.6cm} ISBN: 970-32-4070-4

\vskip6pt %10pt
{\bfseries TEMAS DE MATEMÁTICAS PARA BACHILLERATO}

\vskip3pt %5pt
\bibl{\hbox{\hskip3.5pt}1}{\itshape Cuando cuentes cuántos\dots}
H. A. Rincón Mejía 
1a. Reimpresión de la 2a. edición (2009).
%2a. edición (2006).
%(2002). %124~p.

\bibl{\hbox{\hskip3.5pt}2}{\itshape Sistemas de ecuaciones y de desigualdades.}
A. I. Ramírez-Galarza.
1a. Reimpresión de la 1a. Edición (2008).
%(2002). %116~p.

\bibl{\hbox{\hskip3.5pt}3}{\itshape La historia de un empujón:
un vistazo a las ecuaciones diferenciales ordinarias 
y a los sistemas dinámicos.}\\
L.$\!$ Ortiz Bobadilla y E.$\!$ Rosales González.
3a. reimpresión de la 1a. edición (2017).
%(2007).
%(2002). %178~p.

\bibl{\hbox{\hskip3.5pt}4}{\itshape Dos o tres trazos.}
S. Cárdenas Rubio.
2a. edición (2008).\\
(La nueva edición de este título se encuentra en la colección {\itshape papirhos})

%(2003). %109~p. 

\bibl{\hbox{\hskip3.5pt}5}{\itshape Estadística descriptiva para bachillerato.}
M.\,P. Alonso Reyes, J.\,A. Flores Díaz
(2004). % ~p.

\bibl{\hbox{\hskip3.5pt}6}{\itshape Relaciones de equivalencia.}
M. Cruz Terán.
(2006). % ~p.

\bibl{\hbox{\hskip3.5pt}7}{\itshape Mosaicos.}
L. Hidalgo.
(2007). % ~p.

\bibl{\hbox{\hskip3.5pt}8}{\itshape Funciones Circulares.}
M. Cruz Terán.
(2008). % ~p.

\vskip10pt %10pt
{\bfseries PAPIRHOS}\\[5pt]
\hbox{\hskip.6cm}{\bfseries Serie: ICOSAEDRO}

\vskip3pt %5pt
\bibl{\hbox{\hskip3.5pt}1}{\itshape Dos o tres trazos.}
S. Cárdenas Rubio. 2a. Edición (2017) 

\bibl{\hbox{\hskip3.5pt}2}{\itshape Cónicas, cuádricas y aplicaciones.}
A. I. Ramírez-Galarza. 1a. Edición (2015)\\[10pt]
{\bfseries Serie: TEXTOS}

\vskip3pt %5pt
\bibl{\hbox{\hskip3.5pt}1}{\itshape Grupos I.}
D. Avella Alaminos, O. Mendoza Hernández, E.\,C. Sáenz Valadez,
M.\,J. Souto Salorio.\\
3a. Edición (2017) 

\bibl{\hbox{\hskip3.5pt}2}{\itshape Análisis Matemático.}
M. Clapp. 2a. Edición (2017).

\bibl{\hbox{\hskip3.5pt}3}{\itshape Topología Diferencial.}
V. Guillemin, A. Pollack. 1a. Edición (2015)

\bibl{\hbox{\hskip3.5pt}4}{\itshape Grupos II.}
D. Avella Alaminos, O. Mendoza Hernández, E.\,C. Sáenz Valadez,
M.\,J. Souto Salorio.\\
1a. Edición (2016) 

\bibl{\hbox{\hskip3.5pt}5}{\itshape Geometría euclidiana bidimensional y su grupo de transformaciones.}
M. Cruz López, M. García Campos.\\
1a. Edición (2017).

\bibl{\hbox{\hskip3.5pt}6}{\itshape Curso introductorio de Álgebra I.}
D. Avella Alaminos, G. Campero Arenas.
1a. Edición (2017).

%\vfill\eject %\vskip15pt
%$\hrulefill$
%\vskip1pt %5pt
%$\hrulefill$

\vskip10pt 

{\bfseries CUADERNOS DE OLIMPIADAS DE MATEMÁTICAS}

\vskip3pt %5pt
\bibl{\hbox{\hskip3.5pt}1}{\itshape Combinatoria.}
M. L. Pérez-Seguí.
2a. edición (2016) 
%3a. Reimpresión de la 3a. edición (2010).
%1a. Reimpresión de la 3a. edición (2007).
%3a. edición (2005).
%2a. Reimpresión de la 2a. edición (2004).
%1a. Reimpresión de la 2a. edición (2003). 
%2a. edición (2002).
%(2000). %133~p.

\bibl{\hbox{\hskip3.5pt}2}{\itshape Principios de olimpiada.} 
A. Illanes Mejía. 
2a. edición (2017)
%1a. edición (2002)
%6a. Reimpresión (2011).
%5a. Reimpresión de la 1a. edición (2008).%3a. Reimpresión de la 2a. edición (2008).
%4a. Reimpresión de la 1a. edición (2006).%2a. Reimpresión de la 2a. edición (2006).
%3a. Reimpresión de la 1a. edición (2004).%1a. Reimpresión de la 2a. edición (2004).
%2a. reimpresión de la 1a. edición (2003). %2a. edición (2003).
%1a. reimpresión de la 1a. edición (2002).
%(2001). %120~p.

\bibl{\hbox{\hskip3.5pt}3}{\itshape Geometría.}
R. Bulajich, J. A. Gómez Ortega.
2a. edición (2016)
%9a. Reimpresión (2010).
%6a. Reimpresión (2009).
%5a. Reimpresión (2007).
%4a. Reimpresión (2006).
%3a. Reimpresión (2004).
%2a. Reimpresión (2004).
%1a. Reimpresión (2003). 
%(2003). %188~p.

\bibl{\hbox{\hskip3.5pt}4}{\itshape Geometría. Ejercicios y problemas.}
R. Bulajich, J. A. Gómez Ortega.
2a. edición (2016)
%6a. Reimpresión (2010).
%5a. Reimpresión (2008).
%4a. Reimpresión (2006).
%3a. Reimpresión (2005).
%2a. Reimpresión (2004).
%1a. Reimpresión (2003).
%(2003). %147~p.

\bibl{\hbox{\hskip3.5pt}5}{\itshape Teoría de números.}
M. L. Pérez-Seguí.
2a. edición (2016)
%6a. Reimpresión (2011).
%2a. Edición (2009).
%4a. Reimpresión (2008).
%3a. Reimpresión (2006).
%2a. Reimpresión (2004).
%1a. Reimpresión (2004).
%(2003).
%138~p.

\bibl{\hbox{\hskip3.5pt}6}{\itshape Desigualdades.}
R. Bulajich, J.\,A. Gómez Ortega, R. Valdez Delgado. 
5a. edición (2016)
%4a. edición (2010).
%3a. edición (2007).
%2a. edición (2005).
%(2004). %132~p.

\bibl{\hbox{\hskip3.5pt}6a}{\itshape Inequalities.}
R. Bulajich, J.\,A. Gómez Ortega 
1a. edición (2005).
%(2005) %192~p.
(Traducción del libro ``Desigualdades").

\bibl{\hbox{\hskip3.5pt}7}{\itshape Olimpiadas en SLP, elemental.} 
R. Bulajich, C. J. Rubio 
2a. edición (2016)
%(2011). % ~p.

\bibl{\hbox{\hskip3.5pt}8}{\itshape Olimpiadas en SLP, avanzado.} 
R. Bulajich, C. J. Rubio 
2a. edición (2017)
%1a. edición (2012)
%(2012) % ~p.

%\bibl{\hbox{\hskip3.5pt}8}{\itshape Las olimpiadas de matemáticas en San Luis Potosí, 1987--2005.} 
%R. Bulajich, C. J. Rubio 
%(2008). % ~p.

\bibl{\hbox{\hskip3.5pt}9}{\itshape Matemáticas preolímpicas.}
M. L. Pérez-Seguí.
2a. edición (2017).
%3a. Reimpresión (2011).
%1a. Reimpresión (2009).
%(2008). % ~p.

\bibl{\hbox{\hskip3.5pt}10}{\itshape Combinatoria para olimpiadas internacionales.}
P. Soberón.
2a. edición (2017)
%1a. edición (2010)
%(2010).

\bibl{\hbox{\hskip3.5pt}11}{\itshape Problemas avanzados de olimpiada.}
A. Alberro, R. Bulajich, C. J. Rubio
2a. edición (2017)
%1a. edición (2010)
%(2010).

\bibl{\hbox{\hskip3.5pt}12}{\itshape Combinatoria avanzada.}
M. L. Pérez-Seguí. 
1a. edición (2010)
%(2010).

\bibl{\hbox{\hskip3.5pt}13}{\itshape Principio de las casillas.}
J.\,A. Gómez Ortega, R. Valdez Delgado, R. Vázquez Padilla.\\
3a. edición (2017).
%(2011).

\bibl{\hbox{\hskip3.5pt}14}{\itshape Álgebra.}
R. Bulajich, J.\,A. Gómez Ortega, R. Valdez Delgado.
3a. edición (2016)
%(2014).

\vskip10pt %20pt
{\bfseries APORTACIONES MATEMÁTICAS}

\vskip6pt %15pt
\hskip.4cm {\bfseries Serie: INVESTIGACIÓN}

\bibl{\hbox{\hskip3.5pt}1}{\itshape Coloquio de sistemas dinámicos. Memorias. Guanajuato,
México, 1983.} 
Editado por J.$\!$~A. Seade, G. Sienra (1985). %167~p.

\bibl{\hbox{\hskip3.5pt}2}{\itshape Categorical topology - The complete work of Graciela
Salicrup.} 
Edited by H. Herrlich, C. Prieto (1988). %409~p.

\bibl{\hbox{\hskip3.5pt}3}{\itshape Sistemas dinámicos holomorfos en superficies.} 
X. Gómez-Mont, L. Ortíz-Bobadilla. 
2a. edición (2004). %(1989). %207~p.

\bibl{\hbox{\hskip3.5pt}4}{\itshape Simposio de probabilidad y procesos estocásticos.
Memorias. Guanajuato, México 1988.} \\
Editado por M.$\!$~E. Caballero, L.$\!$~G. Gorostiza (1989).
%136~p. 

\bibl{\hbox{\hskip3.5pt}5}{\itshape Topics in algebraic geometry. Proceedings.
Guanajuato, México 1989.} \\
Editado por L. Brambila-Paz, X. Gómez-Mont (1992). %120~p.

\bibl{\hbox{\hskip3.5pt}6}{\itshape Seminario internacional de álgebra y sus aplicaciones.
Memorias. México 1991.} \\
Editado por L.$\!$~M. Tovar, C. Rentería, R.$\!$~H. Villarreal
(1992). %266~p. 

\bibl{\hbox{\hskip3.5pt}7}{\itshape II Simposio de probabilidad y procesos estocásticos. I
Encuentro México-Chile de análisis estocástico. Memorias.
Guanajuato, México 1992.} 
Editado por M.$\!$~E. Caballero, L.$\!$~G. Gorostiza (1992). %191~p.

\bibl{\hbox{\hskip3.5pt}8}{\itshape Taller de geometría diferencial sobre espacios de
geometrías. Memorias. Guanajuato, México 1992.} \\
Editado por L. del Riego, C.$\!$~T.$\!$~J. Dodson (1992). %113~p.

\bibl{\hbox{\hskip3.5pt}9}{\itshape Poblaciones aleatorias ramificadas y sus equilibrios.}
A. Wakolbinger (1994). %88~p. 

\bibl{10}{\itshape Una Introducción a la geometría computacional a
través de los teoremas de la galería de arte.} \\
V. Estivill-Castro (1994). %51~p.

\bibl{11}{\itshape III Simposio de probabilidad y procesos estocásticos.
Memorias. Hermosillo, México 1994.} \\
Editado por M.$\!$~E. Caballero, L.$\!$~G. Gorostiza (1994). %183~p.

\bibl{12}{\itshape IV Simposio de probabilidad y procesos estocásticos.
Memorias. Guanajuato, México 1996.} \\
Editado por L.$\!$~G. Gorostiza, J.$\!$~A. León, J.$\!$~A.
López-Mimbela (1996). %168~p. 

\bibl{13}{\itshape Taller de variedades abelianas y funciones theta.
Memorias. Morelia, Mich., México 1996.} \\
Editado por R. Rodríguez, J.$\!$~M. Muñoz Porras, S. Recillas
(1998). %155~p. 

\bibl{14}{\itshape Modelos estocásticos.}
Editado por J. M. González Barrios, L. G. Gorostiza
(1998). %340~p. 

\bibl{15}{\itshape Inverse limits.}
W.$\!$~T. Ingram (2000). %80~p.

\bibl{16}{\itshape Modelos Estocásticos II.}
Editado por D. Hernández, J.$\!$~A. López-Mimbela, R. Quezada
(2001). %293~p.

\bibl{17}{\itshape Topics in infinitely divisible distributions and Lévy
  processes.} 
A. Rocha-Arteaga, K. Sato (2003). %114~p. 

\bibl{18}{\itshape Parametric Optimization and Related Topics VII.} \\
Editado por J. Guddat, H.$\!$~Th. Jongen, J.-J. R\"uckmann, M. Todorov
(2004). %252~p.

\bibl{19}{\itshape Continuum Theory: in Honor of Professor David P. Bellamy on the occasion of this 60th Birthay.} \\
Edited by I. W. Lewis, S. Macías, S. B. Nadler, Jr. (2007). %160 p.

\bibl{20}{\itshape Proceedings of the Fourteenth International
  Conference on Fibonacci Numbers and
  ther Applications.} \\
Editado por Florian Luca, Pantelimon St\u anic\u a (2011).

\vfill\eject
\vskip6pt %15pt
\hskip.4cm {\bfseries Serie: COMUNICACIONES}

\bibl{\hbox{\hskip3.5pt}1}{\itshape Programa de investigación del XVIII congreso nacional
de la Sociedad Matemática Mexicana. Memorias. Mérida,
México 1984.} 
Editadas por M. Clapp, J.$\!$~A. Seade (1986). %397~p.

\bibl{\hbox{\hskip3.5pt}2}{\itshape Teoremas límite de alta densidad para campos
aleatorios ramificados.} 
B. Fernández (1986). %145~p. 

\bibl{\hbox{\hskip3.5pt}3}{\itshape Programa del XIX congreso nacional de la Sociedad
Matemática Mexicana, Vol.~I. Memorias. Guadalajara,
México 1986.} 
Editadas por J.$\!$~A. de la Peña, C. Prieto, G.~Valencia, L. 
Verde (1987). %311~p. 

\bibl{\hbox{\hskip3.5pt}4}{\itshape Programa del XIX congreso nacional de la Sociedad
Matemática Mexicana, Vol. II. Memorias. Guadalajara,
México 1986.} 
Editado por J.$\!$~A. de la Peña, C. Prieto, G.~Valencia, L. Verde
(1987). %320~p. 

\bibl{\hbox{\hskip3.5pt}5}{\itshape Programa del XX congreso nacional de la Sociedad
Matemática Mexicana. Memorias. Xalapa, México 1987.}\\
Editado por M.$\!$~A. Aguilar, L. Salmerón, C. Vargas (1988).
%375~p.

\bibl{\hbox{\hskip3.5pt}6}{\itshape XXI Congreso nacional de la Sociedad Matemática
Mexicana. Memorias. Hermosillo, Sonora 1988.} \\
Editado por F. Aranda, J. Bracho, A. Sánchez Valenzuela, A. Vargas
(1989). %335~p. 

\bibl{\hbox{\hskip3.5pt}7}{\itshape Breve introducción a códigos detectores-correctores
de error.} 
C. Rentería, H. Tapia, W.$\!$~Y. Vélez (1990). %36~p.

\bibl{\hbox{\hskip3.5pt}8}{\itshape XXII Congreso nacional de la Sociedad Matemática
Mexicana. Memorias. Puebla, Puebla 1989.}\\
Editado por P. Barrera, A. Illanes, F. O'Reilly, S. Recillas (1990).
%263~p. 

\bibl{\hbox{\hskip3.5pt}9}{\itshape XXIII Congreso nacional de la Sociedad Matemática
Mexicana. Memorias. Guanajuato, México 1990.}\\ 
Editado por A. García-Máynez, L.$\!$~G. Gorostiza, J.
Ize, M. Mendoza (1991). %224~p. 

\bibl{10}{\itshape La estructura de los dendroides suaves.}
S. Macías Alvarez (1993). %50~p.

\bibl{11}{\itshape XXIV Congreso nacional de la Sociedad Matemática
Mexicana. Memorias. Oaxtepec, Morelos 1991.} \\
Editado por O. Hernández, L. Montejano, B. Rumbos, A.$\!$~A.
Wawrzyñzyk (1992). %230~p.

\bibl{12}{\itshape XXV Congreso nacional de la Sociedad Matemática
Mexicana Vol. I. Memorias. Xalapa, Veracruz 1992.} \\ 
Editado por F. Larrión, A. Olvera, V. Pérez-Abreu, E. Vallejo
(1993). %178~p.

\bibl{13}{\itshape XXV Congreso nacional de la Sociedad Matemática
Mexicana Vol. II. Memorias. Xalapa, Veracruz 1992.} \\
Editado por F. Larrión, A. Olvera, V. Pérez-Abreu, E. Vallejo
(1993). %323~p.

\bibl{14}{\itshape XXVI Congreso nacional de la Sociedad Matemática
Mexicana. Memorias. Morelia, Mich. 1993.} \\
Editado por M.$\!$~E. Caballero, J. Delgado, A.$\!$~G. Raggi, J.
Rosenblueth (1994). %487~p.

\bibl{15}{\itshape XI Escuela Latinoamericana de Matemáticas. Memorias.
UNAM, México, D.$\!$ F.; CIMAT, Gto. 1993.} \\
Editado por X. Gómez-Mont, J.$\!$~A. de la Peña, J.$\!$~A. Seade
(1994). %564~p. 

\bibl{16}{\itshape XXVII Congreso nacional de la Sociedad Matemática
Mexicana. Memorias. Querétaro, Qro. 1994.} \\
Editado por J.$\!$~A. León, A. Nicolás, F. Ongay, A. Tamariz
(1995). %536~p. 

\bibl{17}{\itshape Grupo de estudio con la industria y cursos en
matemáticas industriales. Memorias. Oaxaca, Oaxaca 1995.}\\
Editado por A. Fitt, R. Martínez-Villa (1996).
%95~p.

\bibl{18}{\itshape XXVIII Congreso nacional de la Sociedad Matemática
Mexicana. Memorias. Colima, Colima 1995.} \\
Editado por J.$\!$ L. Morales Pérez, S. Pérez Esteva, F.
Sánchez Bringas, G. Villa Salvador (1996). %323~p.

\bibl{19}{\itshape Problemas combinatorios sobre conjuntos finitos de
puntos.} 
B.$\!$~M. Ábrego Lerma (1997). %168~p.

\bibl{20}{\itshape XXIX Congreso nacional de la Sociedad Matemática
Mexicana. Memorias. $\!$San Luis Potosí, SLP~1996.} \\ 
Editado por F. Avila Murillo, R. Montes-de-Oca, R. del Río
Castillo, J. Muciño-Raymundo~(1997). %256~p.

\bibl{21}{\itshape IV Escuela de verano de geometría y sistemas
dinámicos. Memorias. Cimat, Guanajuato 1997.} \\
Editado por O. Calvo, R. Iturriaga (1998). %314~p.

\bibl{22}{\itshape XXX Congreso nacional de la Sociedad Matemática
Mexicana. Memorias. Aguascalientes, Ags. 1997.} \\
Editado por A. López Mimbela, M. Neumann, M. Rzedowski, M.
Shapiro (1998). %361~p.

\bibl{23}{\itshape Segundo grupo de estudio con la industria y cursos 
en matemáticas industriales. Memorias. Cocoyoc, Mor., México. 
1997.}
Editado por A. Fitt, R. Martínez-Villa, H. Ockendon (1999). %98~p.

\bibl{24}{\itshape 3rd. International conference on approximation and
optimization in the Caribean. Proceedings. Puebla, México. 1995.}
Editado por B. Bank, J. Bustamante, J. Guddat, M. A.
Jiménez, H. Th. Jongen, W. R\"omisch (1998). %324~p.

\bibl{25}{\itshape XXXI Congreso nacional de la Sociedad
Matemática Mexicana. Memorias. Hermosillo, Son. 1998.} \\
Editado por P. Padilla, R. Quiroga, C. Signoret, A. Soriano (1999).
%385~p. 

\bibl{26}{\itshape Tendencias interdisciplinarias de las
matemáticas.}
Editado por S. Gitler, C. Prieto (2000). %438~p.

\bibl{27}{\itshape XXXII Congreso nacional de la Sociedad
Matemática Mexicana. Memorias. Guadalajara, Jal. 1999.} \\
Editado por L. Hernández Lamoneda, R. Quezada, J. Martínez
Bernal, H. Sánchez Morgado (2000). %352~p. 

\bibl{28}{\itshape Lecturas Básicas en Topología General.}
Editado por L. M. Villegas, A. Sestier, J. Olivares
(2000). %334~p.

\bibl{29}{\itshape XXXIII Congreso nacional de la Sociedad Matemática
Mexicana. Memorias. Saltillo, Coah. 2000.}\\
Editado por J. Alfaro, M. Eudave, J. González Espino-Barros, E.
Pérez Chavela (2001). %326~p.

\bibl{30}{\itshape XXXIV Congreso Nacional de la Sociedad Matemática
Mexicana. Memorias. Toluca, Méx. 2001.}\\
Editado por G. Contreras, C. Rentería, E. R. Rodríguez,
C. Villegas Blas (2002). %284~p. 

\bibl{31}{\itshape Tópicos de Geometría Algebraica.} 
Editado por L. Brambila, P.\,L. del Angel, A. García Zamora, 
J. Muciño (2002). %210~p. 

\bibl{32}{\itshape XXXV Congreso Nacional de la Sociedad Matemática
Mexicana. Memorias. Durango, Dgo. 2002.} \\
Editado por M. Aguilar, R. Quiroga (2003). %p.

\bibl{33}(Número cancelado.)

\bibl{34}{\itshape XXXVI Congreso Nacional de la Sociedad Matemática
Mexicana. Memorias. Pachuca, Hgo. 2003.}\\
Editado por M. Aguilar, R. Quiroga (2004). %p.

\bibl{35}{\itshape Memorias de la Sociedad Matemática Mexicana.}
Editado por M. Aguilar, R. Quiroga (2005). %p.

\bibl{36}{\itshape Memorias de la Sociedad Matemática Mexicana.}
Editado por M. Aguilar, R. Quiroga (2006). %p.

\bibl{37}{\itshape Memorias de la Sociedad Matemática Mexicana.}
Editado por M. Aguilar, L. Hernández Lamoneda (2007). %p.

\bibl{38}{\itshape Memorias de la Sociedad Matemática Mexicana.}
Editado por M. Aguilar, L. Hernández Lamoneda (2008). %p.

\bibl{39}{\itshape Modelos en estadística y probabilidad.} \\
Editado por J.\,M. González-Barrios, J.\,A. León, A. Pérez, L.\,A. Rincón, J. Villa (2008). %221 p. 

\bibl{40}{\itshape Memorias de la Sociedad Matemática Mexicana.}
Editado por M. Aguilar, L. Hernández Lamoneda (2009). %p.

\bibl{41}{\itshape  Memorias de la Sociedad Matemática Mexicana.} 
Editado por M. Aguilar, L. Hernández Lamoneda (2010). %121 p.

\bibl{42}{\itshape  Las matemáticas a través de los 50 años de la ESFM del IPN.} \\
Editado por Lino Feliciano Reséndis, Luis Manuel Tovar (2011). %338 p.

\bibl{43}{\itshape  Memorias de la Sociedad Matemática Mexicana.} 
Editado por M. Aguilar, L. Hernández Lamoneda (2011). %104 p.

\bibl{44}{\itshape  Modelos en estadística y probabilidad II.} \\
Editado por J.\,M. González-Barrios M., J.\,A. León Vázquez, J. Villa Morales (2011). %194 p.

\bibl{45}{\itshape  Memorias de la Sociedad Matemática Mexicana.} 
Editado por M. Aguilar, L. Hernández Lamoneda (2012). %104 p. 

\bibl{46}{\itshape  Memorias de la Sociedad Matemática Mexicana.} 
Editado por M. Aguilar, L. Hernández Lamoneda (2013). %104 p. 

\bibl{47}{\itshape  Modelos en estadística y probabilidad III.} \\
Editado por J.\,M. González-Barrios M., J.\,A. León Vázquez,
J. Villa Morales, R.\,A Navarro Cruz (2014). %194 p.

\bibl{48}{\itshape  Memorias de la Sociedad Matemática Mexicana.} 
Editado por M. Aguilar, L. Hernández Lamoneda (2014). %104 p. 

\bibl{49}{\itshape  Memorias de la Sociedad Matemática Mexicana.} 
Editado por M. Aguilar, L. Hernández Lamoneda (2015). %104 p. 

\bibl{50}{\itshape  Memorias de la Sociedad Matemática Mexicana.} 
Editado por M. Aguilar, L. Hernández Lamoneda (2016). %104 p. 

\vskip6pt %15pt
\hskip.4cm {\bfseries Serie: TEXTOS}

\bibl{\hbox{\hskip3.5pt}1}{\itshape Introducción a la topología -- Graciela
Salicrup.} \\
Editado por J. Rosenblueth, C. Prieto.
Nivel medio.  
1a. Reimpresión (1997). %312~p.
%(1993). %312~p. 

\bibl{\hbox{\hskip3.5pt}2}{\itshape Procesos estocásticos.} 
C. Tudor.
Nivel avanzado. 
3a. Edición (2002). %692~p.
%2a. Edición (1997). %598~p.
%(1994). %560~p. 

\bibl{\hbox{\hskip3.5pt}3}{\itshape Lectures on continuous-time Markov control processes.}
O. Hernández-Lerma. 
Nivel avanzado
(1994). %67~p.

\bibl{\hbox{\hskip3.5pt}4}{\itshape Un curso de lógica matemática.} 
C.$\!$~R. Videla. 
Nivel avanzado 
(1995). %250~p.

\bibl{\hbox{\hskip3.5pt}5}{\itshape Rudimentos de mansedumbre y salvajismo en
teoría de representaciones.} 
F. Larrión, A.$\!$~G. Raggi, L. Salmerón. 
Nivel avanzado
(1995). %239~p.

\bibl{\hbox{\hskip3.5pt}6}{\itshape Teoría general de procesos e integración
estocástica.} 
T. Bojdecki. 
Nivel avanzado.
1a. Reimp.~(2004).
%(1995). %261~p.

\bibl{\hbox{\hskip3.5pt}7}{\itshape Intersection theory.} 
S.$\!$~X. Descamps. 
Nivel avanzado 
(1996). %122~p.

\bibl{\hbox{\hskip3.5pt}8}{\itshape Inverse problems.} 
H.$\!$~W. Engl. 
Nivel avanzado
(1996). %87~p.

\bibl{\hbox{\hskip3.5pt}9}{\itshape El ABC de los splines.} 
P. Barrera, V. Hernández, C. Durán. 
Nivel elemental
(1996). %245~p.

\bibl{10}{\itshape Lo antiguo y lo nuevo acerca de los conjuntos
convexos.} 
H. Hadwiger. 
Traducción: L. Montejano.
Nivel medio 
(1998). %164~p.

\bibl{11}{\itshape Matemáticas para las ciencias naturales.}
J.$\!$~L. Gutiérrez Sánchez, F. Sánchez Garduño.
Niveles medio y avanzado 
(1998). %590~p. 

\bibl{12}{\itshape Introducción a la teoría de redes.}
M.$\!$~C. Hernández Ayuso. 
Nivel medio
2a. Edición
(2005).
%(1997). %259~p.

\bibl{13}{\itshape Teoría de conjuntos (una introducción).}
F. Hernández Hernández.
Nivel medio
1a. Edición
(2017).
%2a. Edición
%(2003).
%(1998). %342~p.

\bibl{14}{\itshape Lectures on quantum probability.}
A. M. Chebotarev. 
Nivel avanzado 
(2000). %292~p.

\bibl{15}{\itshape Construcción de procesos autosimilares con variancia
  finita.}$\!$
J.$\!$ E. Figueroa López.$\!$
Nivel avanzado~(2000). %215~p.

\bibl{16}{\itshape Grupos algebraicos y teoría de invariantes.}
C. Sancho de Salas.
Nivel avanzado
(2001). %394~p.

\bibl{17}{\itshape Cohomología de Galois de campos locales.}
F. Zaldívar.
Nivel avanzado
(2001). %214~p.

\bibl{18}{\itshape Dimension Theory: An Introduction with Exercises.}
S.\,B. Nadler, Jr. 
Nivel avanzado 
(2002). %180~p. 

\bibl{19}{\itshape Cómputo Numérico con aritmética de punto
  flotante IEEE. Con un teorema, una regla empírica y ciento un
  ejercicios.}
M. L. Overton. 
Traducción: A. Casares Maldonado.
Nivel medio 
(2002) %123~p.
Coedición SIAM.

\bibl{20}{\itshape Topología diferencial.}
V. Guillemin, A. Pollack.
Traducción: O.$\!$ Palmas Velasco.
Nivel medio~(2003).\\ %294~p.
(La nueva edición de este título se encuentra en la colección {\itshape papirhos})

\bibl{21}{\itshape Elementos de Probabilidad y Estadística.}
A. Hernández-del-Valle, O. Hernández-Lerma.
Nivel elemental 
(2003). %280~p.

\bibl{22}{\itshape Topología General.} 
D. Hinrichsen, J. L. Fernández Muñiz, A.
Fraguela Collar, Á. Álvarez Prieto. 
Nivel medio 
(2003). %668~p. 

\bibl{23}{\itshape Introducción a los grupos topológicos de
transformaciones.} 
S. de Neymet U. Con la colaboración de R. Jiménez B.
Nivel avanzado 
(2005). %126~p. 

\bibl{24}{\itshape Breviario de teoría analítica de los
números.}
E. P. Balanzario. 
Nivel medio 
1a. Reimpresión
(2009) Coedición Reverté.
%(2003). %158~p.

\bibl{25}{\itshape Cálculo de probabilidades.}
F. M. Hernández Arellano.
Nivel elemental
(2003). %810~p.

\bibl{26}{\itshape Números primos y aplicaciones.}
F. Luca.
Nivel avanzado
(2004).

\bibl{27}{\itshape Historia y desarrollo de la teoría de los
continuos indescomponibles.}
F.\,L. Jones.
Traducción: S. Macías.\\
Nivel medio
(2004).

\bibl{28}{\itshape Hiperespacios de continuos.}
A. Illanes.
Nivel medio
(2004).

\bibl{29}{\itshape Cadenas de Markov.}
M.\,E. Caballero, N.\,S. Hérnández, V.\,M. Rivero, G. Uribe Bravo, C. Velarde.\\
Nivel medio
1a. Edición
(2017).
%(2004).

\bibl{30}{\itshape The fixed point property for continua.}
S.\,B. Nadler, Jr. 
Nivel avanzado 
(2005).

\bibl{31}{\itshape Invitación a la teoría de los continuos y sus 
  hiperespacios.}
Editado por R. Escobedo, S. Macías, H. Méndez.
Nivel medio
(2006).

\bibl{32}{\itshape Introducción a la teoría de grupos.}
F. Zaldívar. 
Nivel medio
1a. Reimpresión
(2009). Coedición Reverté.
%(2006). Coedición Reverté.

\bibl{33}{\itshape Hyperspaces of sets. A text with research questions.}
S.\,B. Nadler, Jr.
Nivel avanzado
(2006).

\bibl{34}{\itshape Introducción a la optimización no lineal.}
E. Accinelli.
Nivel avanzado
(2009). Coedición Reverté.

\bibl{35}{\itshape Graphs, rings and polyhedra.}
I. Gitler, R. H. Villarreal.
Nivel avanzado
(2011).

\bibl{36}{\itshape Introducción a la topología de conjuntos.} 
A. García Máynez. Nivel elemental (2011). %143 p.

\bibl{37}{\itshape Elementos de topología general.}
F. Casarrubias Segura, A. Tamariz Mascarúa.\\
 Nivel medio 3a. edición (2016). (Por aparecer) %348 p.

\bibl{38}{\itshape Cálculo.}
H. Arizmendi, A. Carrillo, M. Lara. Nivel elemental 2a. edición (2016).

\bibl{39}{\itshape Curso elemental de probabilidad y estadística.}
Luis Rincón.  Nivel elemental (2013). %348 p.

\vskip10pt
{\itshape Alberto Barajas: Su oratoria, sus matemáticas y sus enseñanzas.}\\
\hbox{\hskip.6cm} 
Edición: V. Neumann-Lara,
I. Puga, S. Macías 
(2010)  
346 p. Contiene 2 DVD.

\vskip10pt
%\hskip.6cm  {\bfseries LIBRARIA}
%
%\hskip.6cm  {\itshape Matemáticas e imaginación.} 
%E. Kasner, J. Newman. 
%$\!$Coedición Libraria-SMM (2007), 256 p.
{\bfseries PUBLICACIONES DEL INSTITUTO DE MATEMÁTICAS EN COLABORACIÓN CON APORTACIONES MATEMÁTICAS}
\vskip10pt
\hskip.6cm {\bfseries Contemporary Mathematics}

\bibl{260}{\itshape First summer school in analysis and mathematical
  physics. Cuernavaca, Morelos, 1998.}
Editado por S. Pérez-Esteva, C. Villegas-Blas. 
Contemporary Mathematics No. 260. 
Coedición de Aportaciones Matemáticas -- American Mathematical
Society (2000). %132~p.

\bibl{289}{\itshape Second summer school in analysis and mathematical
  physics. Cuernavaca, Morelos, 2000.}
Editado por S. Pérez-Esteva, C. Villegas-Blas. 
Contemporary Mathematics No. 289. 
Coedición de Aportaciones Matemáticas -- American Mathematical
Society (2001). %272~p.

\bibl{336}{\itshape Stochastic Models.}
Editado por J.\,M. González-Barrios, J.\,A. León, A. Meda. 
Contemporary Mathematics No. 336. 
Coedición de Aportaciones Matemáticas -- American Mathematical
Society (2003). %272~p.

\bibl{340}{\itshape Spectral Theory of Schr\"odinger Operators.
  Lecture Notes from a Workshop on Schr\"odinger Operator
  Theory. IIMAS, UNAM, Mexico, 2001.}
Editado por R. del Río, C. Villegas-Blas.
Contemporary Mathematics No. 340
Coedición de Aportaciones Matemáticas -- American Mathematical
Society (2004). %249~p.

\bibl{341}{\itshape Topological algebras and their applications. 
  Fourth International Conference on Topological Algebras
  and Their Applications. Oaxaca, Mexico, 2002.}
Editado por H. Arizmendi, C. Bosch, L. Palacios.
Contemporary Mathematics No. 341.
Coedición de Aportaciones Matemáticas -- American 
Mathematical Society (2004). %137~p.

\bibl{389}{\itshape Geometry and Dynamics International Conference in
  Honor of the 60th Anniversary of Alberto Verjovsky. January 6-11, 
  2003. Cuernavaca, Mexico.}
Editado por J. Eells, E. Ghys, M. Lyubich, J. Palis,
  J. Seade. 
Contemporary Mathematics No. 389.
Coedición de Aportaciones Matemáticas -- American Mathematical
Society (2005). %198 p.

\bibl{476}{\itshape Fourth summer school in analysis and mathematical physics.
Topics in  Spectral Theory and Quantum Mechanics.
Cuernavaca, Morelos, 2005}. 
Editado por C. Villegas-Blas. 
Contemporary Mathematics No. 476. 
Coedición de Aportaciones Matemáticas -- American Mathematical Society (2008).
%148 p.

\vskip15pt\vfill

\parbox{\linewidth}{Información y pedidos:

\vskip5pt
\begin{centering}
  Sección de Publicaciones\\
  Instituto de Matemáticas, UNAM, Circuito Exterior\\
  Ciudad Universitaria, 04510 Ciudad de México, MÉXICO\\
  tel: +52 55 5622-4496 y 4545\\
%  fax: +52 55 5550-1342 y +52 55 5616-0348\\
  e-mail: \texttt{papirhos@im.unam.mx}\\
  e-mail: \texttt{librosdemate@im.unam.mx}\\
  web: \texttt{http://papirhos.matem.unam.mx/}\\
\end{centering}}
\end{document}
