%% 3a. reimpresión 3a. edición (oct2023)
%% 2a. reimpresión 2a. edición (jul2019)
%% 1a. reimpresión 2a. edición (nov2018)
%% 2a. edición
%% 1a. edición
\documentclass[10pt,pagesize]{scrbook}

%%%%%%%%%%%%%%%%%%%%%%%%%%%%%%%%%%%%%%%%%%%%%%%%%%%%%%%%%%%%%%%%%%%%%%%%
% Start settings for papirhos design
%%%%%%%%%%%%%%%%%%%%%%%%%%%%%%%%%%%%%%%%%%%%%%%%%%%%%%%%%%%%%%%%%%%%%%%%

% Layout
%%%%%%%%%%%%%%%%%%%%%%%%%%%%%%%%%%%%%%%%%%%%%%%%%%%%%%%%%%%%%%%%%%%%%%%%

\KOMAoptions{twoside,paper=170mm:230mm,headinclude=true,footinclude=false}
%\areaset[20mm]{120mm}{200mm}
\typearea[15mm]{14}
% \geometry ...
\usepackage{setspace}
\setstretch{1.05}

%\usepackage{scrpage2}
\usepackage{scrlayer-scrpage}
\pagestyle{scrheadings}
\clearscrheadfoot
%\ohead{\pagemark}
\ohead{\textbf{\pagemark}}
\lohead{\headmark}
\rehead{\headmark}
%\lohead{\raisebox{0.15em}\headmark}
%\rehead{\raisebox{0.15em}\headmark}

\addtolength{\headheight}{1mm}


% Extra definitions
%%%%%%%%%%%%%%%%%%%%%%%%%%%%%%%%%%%%%%%%%%%%%%%%%%%%%%%%%%%%%%%%%%%%%%%%

\newcommand{\missing}{\ensuremath{\clubsuit\clubsuit\clubsuit}}

% end xelatex header
%%%%%%%%%%%%%%%%%%%%%%%%%%%%%%%%%%%%%%%%%%%%%%%%%%%%%%%%%%%%%%%%%%%%%%%%

% Font selection
%%%%%%%%%%%%%%%%%%%%%%%%%%%%%%%%%%%%%%%%%%%%%%%%%%%%%%%%%%%%%%%%%%%%%%%%

\usepackage[T1]{fontenc}                % use this if European fonts (ec) package is installed
\usepackage{textcomp}                   % adds some additional symbols
\usepackage[scaled=0.92]{helvet}        % set Helvetica as the sans-serif font
\renewcommand{\rmdefault}{ptm}          % set Times as the default text font

% Uncomment the next line if you do not have the mathematics font
% MTPro2 installed:
\usepackage[mtphrb,mtpfrak,mtpscr,subscriptcorrection,slantedGreek,nofontinfo]{mtpro2}

% Element formatting
%%%%%%%%%%%%%%%%%%%%%%%%%%%%%%%%%%%%%%%%%%%%%%%%%%%%%%%%%%%%%%%%%%%%%%%%

\KOMAoptions{headings=big,open=right,chapterprefix=true}
\addtokomafont{chapter}{\normalfont}
\addtokomafont{chapterentry}{\normalfont\bfseries}
\addtokomafont{chapterprefix}{\normalfont\bfseries}
\addtokomafont{section}{\normalfont\bfseries}
\addtokomafont{subsection}{\normalfont\bfseries}
\addtokomafont{subsubsection}{\normalfont\bfseries}
\addtokomafont{paragraph}{\normalfont\bfseries}
\addtokomafont{subparagraph}{\normalfont\bfseries}
\addtokomafont{part}{\normalfont}
\addtokomafont{partnumber}{\normalfont\bfseries}
\renewcommand*{\partformat}{Parte~\thepart}

\renewcommand{\chapterheadstartvskip}{\vspace*{-1.85\baselineskip}}
\renewcommand{\chapterheadendvskip}{\vspace*{9\baselineskip
    plus .5\baselineskip minus .5\baselineskip}}

\renewcommand*{\chapterformat}{%
\mbox{\thechapter}}
\renewcommand*{\chaptermarkformat}{%
\thechapter\autodot\enskip}

\newcommand{\mathimagefontsize}{\footnotesize}

\makeatletter
\let\originall@part\l@part
\def\l@part#1#2{\originall@part{{\normalfont\bfseries%
      #1}}{\normalfont\bfseries #2}}
\makeatother

% Language selection
%%%%%%%%%%%%%%%%%%%%%%%%%%%%%%%%%%%%%%%%%%%%%%%%%%%%%%%%%%%%%%%%%%%%%%%%

\usepackage[utf8]{inputenc}
%\usepackage[spanish,mexico,es-preindex,es-noindentfirst]{babel}
\usepackage[spanish,mexico,es-noindentfirst]{babel}
\usepackage[spanish=spanish]{csquotes}
\usepackage[inline]{enumitem}
\usepackage{amsmath}
\usepackage{amsthm}
\usepackage{upref}
\usepackage{exscale}
\usepackage[leqno]{mathtools}
\usepackage{multicol}

% Bibliography
%%%%%%%%%%%%%%%%%%%%%%%%%%%%%%%%%%%%%%%%%%%%%%%%%%%%%%%%%%%%%%%%%%%%%%%%
%\DeclareFieldFormat[article]{pages}{#1}

% Processing options
%%%%%%%%%%%%%%%%%%%%%%%%%%%%%%%%%%%%%%%%%%%%%%%%%%%%%%%%%%%%%%%%%%%%%%%%

% Esto está porque pdftk (versión 1.44-1.45) tiene problemas con pdf
% 1.5.  Despues de instalar pdftk con un parche
% (http://bugs.debian.org/cgi-bin/bugreport.cgi?bug=687669) podemos
% quitar la restricción de sólo usar PDF 1.4.
%\pdfminorversion=4

\usepackage[all]{nowidow}
\usepackage{microtype}
\usepackage{scrhack}

\usepackage{imakeidx}
\makeindex[title=Índice analítico, columnsep=20pt, intoc]
\indexsetup{headers={\indexname}{\indexname}, othercode={\small\setstretch{1.115}}}
\makeatletter
\newif\iffirst@subitem
\def\@idxitem{%
  \par\hangindent40\p@ % original
  \first@subitemtrue   % added
}
\def\subitem{%
  \par\hangindent40\p@
  \iffirst@subitem
    \nobreak
    \first@subitemfalse
  \fi
  \hspace*{20\p@}}
\makeatother
%\addto\captionsspanish{\renewcommand\indexname{Índice alfabético}%
\def\bibname{Referencias}

\usepackage{float}
% This puts floats top aligned on float pages:
\makeatletter
    \setlength\@fptop{0\p@}
    \setlength\@fpsep{11\p@}
\makeatother

\allowdisplaybreaks

% Formatting, numbering
%%%%%%%%%%%%%%%%%%%%%%%%%%%%%%%%%%%%%%%%%%%%%%%%%%%%%%%%%%%%%%%%%%%%%%%%

\numberwithin{equation}{chapter}
\setcounter{tocdepth}{1} % TODO: esto todavía no está decidido

% captions TODO: revisar
%%%%%%%%%%%%%%%%%%%%%%%%%%%%%%%%%%%%%%%%%%%%%%%%%%%%%%%%%%%%%%%%%%%%%%%%

%\usepackage[margin=\parindent,font=small,labelfont=bf,
%labelsep=colon,labelformat=simple]{caption}
%
%\usepackage[labelformat=simple]{subcaption}
%\renewcommand\thesubfigure{\textup{(\alph{subfigure})}}%
%
%%%%%%%%%%%%%%%%%%%%%%%%%%%%%%%%%%%%%%%%%%%%%%%%%%%%%%%%%%%%%%%%%%%%%%%%
% End settings for papirhos design
%%%%%%%%%%%%%%%%%%%%%%%%%%%%%%%%%%%%%%%%%%%%%%%%%%%%%%%%%%%%%%%%%%%%%%%%

%%%%%%%%%%%%%%%%%%%%%%%%%%%%%%%%%%%%%%%%%%%%%%%%%%%%%%%%%%%%%%%%%%%%%%%%
% Start individual settings
%%%%%%%%%%%%%%%%%%%%%%%%%%%%%%%%%%%%%%%%%%%%%%%%%%%%%%%%%%%%%%%%%%%%%%%%

%\usepackage{graphicx}

\usepackage{tikz}
% \tikzset{font=\small}
% \tikzset{every node/.append style={transform shape=false}}
% \makeatletter
% \tikzset{ScalePlotMarksOff/.code={
%         \def\pgfuseplotmark##1{\pgftransformresetnontranslations\csname pgf@plot@mark@##1\endcsname}
% }}
% \makeatother

\usepackage{pgfplots}
\pgfplotsset{compat=1.10}
\usetikzlibrary{arrows,positioning}
\tikzset{font=\small}
\tikzset{every node/.append style={transform shape=false}}
\makeatletter
\tikzset{ScalePlotMarksOff/.code={
    \def\pgfuseplotmark##1{\pgftransformresetnontranslations\csname%
      pgf@plot@mark@##1\endcsname}
  }}
\makeatother

\makeatletter
 \pgfplotsset{
    xtick parsed/.code={
        \c@pgf@counta 0\relax
        \foreach \x in {#1} {
            \pgfmathparse{\x}
            \ifnum\c@pgf@counta=0
                \xdef\pgfplots@xtick{\pgfmathresult}
            \else
                \xdef\pgfplots@xtick{\pgfplots@xtick,\pgfmathresult}
            \fi
            \global\advance\c@pgf@counta 1\relax
        }
    }
 } 
 \makeatother

%%%%%%%%%%%%%%%%%%%%%%%%% Extra Definitions %%%%%%%%%%%%%%%%%%%%%%%%%

% this creates an \item* command for starred exercises

\makeatletter
\def\makestarredlabel#1{%
\settowidth\@tempdima{#1}%
\global\let\makelabel\oldmakelabel
\makelabel{\hbox to\@tempdima{\llap{*\kern.2em}#1}}%
}
\let\olditem\item
\def\item{\@ifstar\starreditem\olditem}
\def\starreditem{%
\let\oldmakelabel\makelabel
\let\makelabel\makestarredlabel
\olditem
}
\makeatother

\newtheorem{theorem}{Teorema}[chapter]
\newtheorem{corollary}[theorem]{Corolario}
\newtheorem{definition}[theorem]{Definición}
\newtheorem{example}[theorem]{Ejemplo}
\newtheorem{exercise}[theorem]{Ejercicio}
\newtheorem{lemma}[theorem]{Lema}
\newtheorem{notation}[theorem]{Notación}
\newtheorem{problem}[theorem]{Problema}
\newtheorem{proposition}[theorem]{Proposición}
\newtheorem{remark}[theorem]{Observaciones}
\newtheorem{obs}[theorem]{Observación}

\newcommand{\vspaceindex}{\vphantom{\int_0^0}}
\newcommand{\FRAME}{\hbox{\missing{} Aquí estuvo una imagen}}

%%%%%%
% Para las notas a pie de página
%%%%%%

\deffootnote[2em]{2em}{1em}{\textsuperscript{\thefootnotemark\ }}

%%%%%%%%%%%%%%%%%%
% Comandos para mca
%%%%%%%%%%%%%%%%%%

\usepackage{newfloat}
\DeclareFloatingEnvironment[
placement=H,
within=none,
]{bio}

\newcommand{\disc}{\ensuremath{\mathrm{disc}}}
\newcommand{\loc}{\ensuremath{\text{\normalfont loc}}}
\DeclareMathOperator{\sen}{sen}
\DeclareMathOperator{\id}{id}
\DeclareMathOperator{\Int}{int}
\DeclareMathOperator{\cerr}{cerr}
\DeclareMathOperator{\dist}{dist}
\DeclareMathOperator{\sop}{sop}
\DeclareMathOperator{\graf}{graf}
\DeclareMathOperator{\vol}{vol}
\DeclareMathOperator{\lin}{lin}

\makeatletter
\newcommand{\opnorm}{\@ifstar\@opnorms\@opnorm}
\newcommand{\@opnorms}[1]{%
  \left|\mkern-1.5mu\left|\mkern-1.5mu\left|
   #1
  \right|\mkern-1.5mu\right|\mkern-1.5mu\right|
}
\newcommand{\@opnorm}[2][]{%
  \mathopen{#1|\mkern-1.5mu#1|\mkern-1.5mu#1|}
  #2
  \mathclose{#1|\mkern-1.5mu#1|\mkern-1.5mu#1|}
}
\makeatother

%%%%%%%%%%%%%%%%%%%%%%%%%%%%%%%%%%%%%%%%%%%%%%%%%%%%%%%%%%%%%%%%%%%%%%%%
% End personal settings
%%%%%%%%%%%%%%%%%%%%%%%%%%%%%%%%%%%%%%%%%%%%%%%%%%%%%%%%%%%%%%%%%%%%%%%%

\newcommand\blfootnote[1]{%
  \begingroup
  \renewcommand\thefootnote{}\footnote{#1}%
  \addtocounter{footnote}{-1}%
  \endgroup
}
\begin{document}
\frontmatter\pagestyle{empty}

\chapter{Colección \emph{papirhos}}

Desde 1985, el Instituto de Matemáticas ha editado distintas series de
libros: \emph{Temas de Bachillerato}, \emph{Aportaciones Matemáticas}
y \emph{Cuadernos de Olimpiadas}; las dos últimas en colaboración con
la Sociedad Matemática Mexicana
.  

La colección \emph{papirhos} abarca tres ámbitos de la
producción matemática: la investigación, la enseñanza y la
difusión. Contiene material cuidadosamente seleccionado en diferentes
niveles y distribuido en seis series: \emph{Mixbaal},
\emph{Icosaedro}, \emph{Textos}, \emph{Notas},
\emph{Monografías} y \emph{Actas}. Además, los libros de \emph{papirhos} serán
editados eventualmente tanto en forma impresa como electrónica.  Las seis series de
\emph{papirhos} son:
	
\begin{description}
\item[{\raisebox{-.7ex}[0cm][0cm]
    {\includegraphics[height=\baselineskip]{papirhos-figs/mixbaal.pdf}}}]
  Escritos que presentan temas de matemáticas de manera lúdica y
  accesible. Dirigida al lector curioso y de cualquier edad en temas
  de matemáticas.
\vfill
\item[{\raisebox{-1.55ex}[0cm][0cm]
    {\includegraphics[height=2\baselineskip]{papirhos-figs/icosaedro.pdf}}}]
  Libros con material complementario a lo que usualmente se enseña en
  las aulas y que también aportan una visión diferente y creativa,
  reflejando el quehacer de las matemáticas. En esta serie los jóvenes
  lectores encontrarán posibles compañeros de viaje.
\vfill
\item[{\raisebox{-.9ex}[0cm][0cm]
    {\includegraphics[height=1.5\baselineskip]{papirhos-figs/textos.pdf}}}]
  Textos en los que se expone, de manera completa y rigurosa, el
  material sobre un curso o seminario de licenciatura o
  posgrado. Dirigida a estudiantes universitarios, académicos e
  investigadores.
\vfill
\item[{\raisebox{-.95ex}[0cm][0cm]
    {\includegraphics[height=1.2\baselineskip]{papirhos-figs/notas.pdf}}}]
  Notas de cursos avanzados de licenciatura y seminarios de posgrado,
  escritas con sumo cuidado y relacionadas con tópicos de
  investigación actual. Dirigida a estudiantes universitarios,
  académicos e investigadores.
\vfill
\item[{\raisebox{-.8ex}[0cm][0cm]
    {\includegraphics[height=1.4\baselineskip]{papirhos-figs/monografias.pdf}}}]
  Tratados profundos y claros que versan sobre áreas específicas de
  las matemáticas y que están enfocados a la investigación.
\vfill
\item[{\raisebox{-.6ex}[0cm][0cm]
    {\includegraphics[height=1\baselineskip]{papirhos-figs/actas.pdf}}}]
  Esta serie recoge las actas que siguen estricto arbitraje de congresos de investigación en matemáticas.
  
\end{description}

\newpage

\noindent
Como parte del Comité Editorial de \emph{papirhos,} es un gusto
presentar este libro y agradecer a todas las personas que
colaboraron en su edición. Por supuesto, reconocemos en primer lugar
a Mónica Clapp quien, con su vasta experiencia en docencia e
investigación, escribió este libro con gran pasión, esmero y
dedicación. Su incansable compromiso con la formación de estudiantes
se ve claramente reflejado en este texto tan esperado por los
lectores.

Queremos expresar nuestra gratitud a Nils Ackermann, quien estuvo
muy presente en la corrección y la formación inicial del libro, y a
Pablo Rosell quien hizo un trabajo sumamente cuidadoso en las figuras
e imágenes que acompañan el texto, así como en la formación y cuidado
de la edición.

Asimismo, agradecemos a Helena Lluis por participar en el cuidado de
la edición y la corrección de estilo de las páginas preliminares,
así como a Leonardo Espinosa y a Celia Osorio de la Sección de
Publicaciones por el apoyo permanente y comprometido con esta
colección.

Por último, agradecemos al pintor oaxaqueño Fernando Olivera el que nos
haya permitido hacer uso de su hermosa obra “La Luna” que
adorna la portada de este libro, así como a Karla López por su
valioso apoyo.

A todos ellos, ¡nuestro más sincero agradecimiento!


\vspace*{5pt}

\begin{flushright}
  %Nils Ackermann, 
  Aubin Arroyo, Laura Ortiz,
  Martha Takane,\\
  Gerónimo Uribe, Paloma Zubieta.
\end{flushright}
\end{document}
