\begin{document}
\frontmatter\pagestyle{empty}

\chapter{Colección \emph{papirhos}}

Desde 1985, el Instituto de Matemáticas ha editado distintas series de
libros: \emph{Temas de Bachillerato}, \emph{Aportaciones Matemáticas}
y \emph{Cuadernos de Olimpiadas}; las dos últimas en colaboración con
la Sociedad Matemática Mexicana
.  

La colección \emph{papirhos} abarca tres ámbitos de la
producción matemática: la investigación, la enseñanza y la
difusión. Contiene material cuidadosamente seleccionado en diferentes
niveles y distribuido en seis series: \emph{Mixbaal},
\emph{Icosaedro}, \emph{Textos}, \emph{Notas},
\emph{Monografías} y \emph{Actas}. Además, los libros de \emph{papirhos} serán
editados eventualmente tanto en forma impresa como electrónica.  Las seis series de
\emph{papirhos} son:
	
\begin{description}
\item[{\raisebox{-.7ex}[0cm][0cm]
    {\includegraphics[height=\baselineskip]{papirhos-figs/mixbaal.pdf}}}]
  Escritos que presentan temas de matemáticas de manera lúdica y
  accesible. Dirigida al lector curioso y de cualquier edad en temas
  de matemáticas.
\vfill
\item[{\raisebox{-1.55ex}[0cm][0cm]
    {\includegraphics[height=2\baselineskip]{papirhos-figs/icosaedro.pdf}}}]
  Libros con material complementario a lo que usualmente se enseña en
  las aulas y que también aportan una visión diferente y creativa,
  reflejando el quehacer de las matemáticas. En esta serie los jóvenes
  lectores encontrarán posibles compañeros de viaje.
\vfill
\item[{\raisebox{-.9ex}[0cm][0cm]
    {\includegraphics[height=1.5\baselineskip]{papirhos-figs/textos.pdf}}}]
  Textos en los que se expone, de manera completa y rigurosa, el
  material sobre un curso o seminario de licenciatura o
  posgrado. Dirigida a estudiantes universitarios, académicos e
  investigadores.
\vfill
\item[{\raisebox{-.95ex}[0cm][0cm]
    {\includegraphics[height=1.2\baselineskip]{papirhos-figs/notas.pdf}}}]
  Notas de cursos avanzados de licenciatura y seminarios de posgrado,
  escritas con sumo cuidado y relacionadas con tópicos de
  investigación actual. Dirigida a estudiantes universitarios,
  académicos e investigadores.
\vfill
\item[{\raisebox{-.8ex}[0cm][0cm]
    {\includegraphics[height=1.4\baselineskip]{papirhos-figs/monografias.pdf}}}]
  Tratados profundos y claros que versan sobre áreas específicas de
  las matemáticas y que están enfocados a la investigación.
\vfill
\item[{\raisebox{-.6ex}[0cm][0cm]
    {\includegraphics[height=1\baselineskip]{papirhos-figs/actas.pdf}}}]
  Esta serie recoge las actas que siguen estricto arbitraje de congresos de investigación en matemáticas.
  
\end{description}

\newpage

<%=meta['presentacion']%>

\vspace*{5pt}

\begin{flushright}
  %Nils Ackermann, 
  Aubin Arroyo, Laura Ortiz,
  Martha Takane,\\
  Gerónimo Uribe, Paloma Zubieta.
\end{flushright}
\end{document}
